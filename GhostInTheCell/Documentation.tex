\documentclass[a4paper]{mwart}
\usepackage[T1]{fontenc}
\usepackage[utf8]{inputenc}
\usepackage[polish]{babel}
\usepackage{polski}
\usepackage{listings}

% Ozdoba hiperłączy
\usepackage[pdfborder={0 0 0}]{hyperref}
% Styl francuski
\frenchspacing
% Wcięcie
\setlength{\parindent}{0pt}
% Paczki na grafike
\usepackage{graphicx}
\usepackage{subfig}
% Ozdoba zdjęć
\usepackage[nofoot,hdivide={2cm,*,2cm},vdivide={2cm,*,2cm}]{geometry}

\author{Kamil Breczko} 
\title{Bot do gry Ghost in the Cell}
\date{9 grudnia 2017}

\begin{document}

\maketitle
\thispagestyle{empty}

\newpage

\tableofcontents
\thispagestyle{empty}

\newpage
\section{Opis gry}
Gra \emph{"Ghost in the Cell"} rozgrywa się na planszy z dwoma graczami, na której umieszczono fabryki (od 7 do 15 fabryk). Początkowo każdy z graczy posiada po jednej fabryce, reszta fabryk jest neutralna. Fabryka może produkować cyborgi, które mają za zadanie bronić daną fabryke i atakować inne. Celem gry jest wyprodukowanie maksymalnej ilości cyborgów, aby zniszczyć przeciwników.
W każdej turze gracz może podjąć nieograniczoną liczbę ruchów:
\begin{itemize}
 \item przeniesienie cyborgów do zaprzyjaźnionej fabryki;
 \item wysłanie cyborgów do wrogiej fabryki, w celu zniszczenia;
 \item wysłanie bomb, w celu uszkodzenia cyborgów stacjonujących w fabryce oraz zablokowanie produkcji cyborgów na 5 tur; (ilość bomb: 2)
 \item zwiększenie produkcji fabryki kosztem 10 cyborgów;
\end{itemize}

\section{Opis algorytmu}
Rozwiązanie do gry \emph{"Ghost in the Cell"} wykorzystuje drzewa decyzyjne. Algorytm podejmuje ciąg decyzji analizując każdą fabrykę z osobna, czyli każda decyzja z danej fabryki nie zależy od stanu innych fabryk gracza. Aby zwyciężyć, algorytm przyjmuje strategie ofensywną. \\ 
Zastosowane techniki:
\begin{itemize}
 \item Co 50 tur wybierana jest "~największa"~ fabryka, pod względem ilości przebywających tam cyborgów. Po wybraniu wrogiej fabryki, wyszukiwana jest fabryka gracza, która znajduje się najbliżej. Następnie wysyłana jest jedna jednostka bomby, w celu zniszczenia znajdujących się tam cyborgów oraz spowolnienie ich produkcji;
 \item W każdej fabryce gracza jest uruchamiana funkcja oceniająca fabryki przeciwnika, w celu określenia najlepszego celu. Dana funkcja wykorzystuje informacje: odległość, ilość cyborgów w fabryce gracza, ilość cyborgów w fabryce przeciwnika, ilość cyborgów wysłanych do fabryki przeciwnika;


\begin{scriptsize}
\begin{lstlisting}[language=Java]
 rateTarget(){
    rateDistance = 2/(distance(myFactory,enemyFactory));
    sumCyborgs = -cyborgs;
    rateProduction = 10* production;
    for (Troop troop : findTroopsByTargetFactory(id, troops)) {
    	sumCyborgs += (troop.getOwner() == 1) ? troop.getCyborgs() : -troop.getCyborgs();
    }
    result = (sumCyborgs + rateProduction) * rateDistance;    
    if (result > 0) {
            return result * 2;
    }
    return result * 1.5;
  }
\end{lstlisting}
\end{scriptsize}
\end{itemize}
Wykonując ruch w danej turze, uwzględniane są wszystkie fabryki. Z każdej fabryki gracza są wysyłane cyborgi do fabryk przeciwnika, aż ilość zrówna się z zerem. Początkowe cele wybierane są zgodnie z oceną funkcji, a ilość cyborgów zgodnie ze wzorem: (ilość\_cyborgów)/2.
\end{document}
